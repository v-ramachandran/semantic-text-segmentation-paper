\documentclass{sig-alternate}
\usepackage{amsmath,amsfonts}
\usepackage[ruled,vlined]{algorithm2e}

\begin{document}

  \title{Semantic Text Segmentation with Sentence Vectors}

  \numberofauthors{2}
  \author{
  \alignauthor Venketaram Ramachandran\\
  \affaddr{Department of Computer Science}\\
  \affaddr{University of Texas at Austin}\\
  \email{venket@cs.utexas.edu}
  \alignauthor Ankita De\\
  \affaddr{Department of Computer Science}\\
  \affaddr{University of Texas at Austin}\\
  \email{ankitade@cs.utexas.edu}
  }

  \date{8 December 2014}
  \maketitle
  \begin{abstract}
    0. Present a novel approach to cluster documents in a text (possible online)
    1. Approach extends ParagraphVector
    2.
  \end{abstract}

  \category{I.2.7}{Artificial Intelligence}{Natural Language Processing}[Language models, Text analysis] \\
  \category{I.5.4}{Pattern Recognition}{Application}[Text processing]

  \terms{Theory}

  \keywords{neural networks, text segmentation, semantic sentence representation}

  \section{Introduction}
  \section{Background}
  \section{Methodology}
  \subsection{Architecture}
  \subsection{Sentence Representation}
  \subsection{Clustering Algorithm}

  Algorithm 1 describes the proposed clustering algorithm.

  \begin{algorithm}
    \caption{\textsc {Generate-Prototype-Clusters} \label{IR2}}
    \SetKw{KwInitialize}{Initialize:}
    \KwIn{vector conversion function \(\sigma\), similarity threshold \(t\)}
    \KwOut{A set of \{Prototype, Cluster\} pairs. }
    \KwInitialize{Output Set \(\alpha = \{\}\),
    \(prototype_{current}=0\),\(cluster_{current}=\{\} \)} \\
    \Begin{
    \For{each answer input a}{
    \(v=\sigma(a)\) \\
    \(\phi=\)cosine similarity between v and \(prototype_{current}\) \\
    \If{\(\phi < t\)}{
    Add v to \(cluster_{current}\) \\
    Recompute \(prototype_{current}\) as centroid of
    \(cluster_{current}\)}
    \Else{
    Add \{\(prototype_{current},cluster_{current}\)\} to output set.\\
    \(prototype_{current}=v\) \\
    \(cluster_{current}=\{v\}\)
    }
    }
    \Return \(\alpha\)
    }
  \end{algorithm}

  \subsection{Intuition}
  \subsection{Online Application}
  \section{Experimental Setup}
  \subsection{Datasets}
  \subsection{Comparison Technique}
  \subsection{Evaluation Methodology}
  \subsubsection{\(P_k\)}
  \subsubsection{WindowDiff}
  \section{Results}
  \subsection{Discussion}
  \section{Future Work}
  \section{Conclusion}
\end{document}
